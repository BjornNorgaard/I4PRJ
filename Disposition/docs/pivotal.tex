\section{Fremlæggelse}

\subsection{Pivotal}

% vi skulle bruge et scrumværktøj
Som et mål i projektet skulle vi bruge et projektstyringsværktøj egnet til arbejde med scrum. Hertil valgte vi pivotal tracker.

% icebox, backlog, current, done
Privotal bruger fire primære 'kasser'.

\begin{table}[H]
	\begin{tabular}{ll}
		\textbf{Icebox}	& Indeholder alle features som muligvis skal implementeres.\\
		\textbf{Backlog}& Ting som er igang, men  ikke er højt nok prioriteret til \textit{current}.\\
		\textbf{Current}& Disse ting kan nås i en sprint, forsat at vi holder nuværende \textit{velocity.}\\
		\textbf{Done}	& Færdiggjorte features.\\
	\end{tabular}
\end{table}

% userstories til tasks
Disse userstories er efterfølgende blevet bruft ned til at være flere mindre \textit{tasks} disse er så igennem forløbet blevet implementeret.

\subsection{Iterativ proces}

% hvordan det ledte til en iterativ process
Da vi har arbejdet med userstories og scrum har vi i høj grad arbejdet iterativt. Dette er sket ved at vi har implementeret én userstory af gangen. 

% klar til release (master) 
Dette har ledt til at vi under det meste af forløbet har haft et produkt som har været klar til realease, dog uden alle features.