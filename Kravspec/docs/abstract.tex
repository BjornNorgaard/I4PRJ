% Initial setup
\markboth{RESUMÉ/ABSTRACT}{RESUMÉ/ABSTRACT}
\abstractintoc
\abstractcol
\setlength{\abstitleskip}{-18pt}

% Abstract
\begin{abstract}
Rapporten omhandler et 3. semesterprojekt fra Ingeniørhøjskolen Aarhus Universitet udarbejdet med henblik på at producere et system, som kan fungere som dosering- og sorteringsanlæg til tabletter. Systemet er tiltænkt brug i ældreplejen, som en måde hvorpå fejlmedicinering kan undgås, og tabletsortering kan klares automatisk. Ydermere vil systemet kunne gøre personale og pårørende opmærksomme på, at brugeren ikke har afhentet sine tabletter, indenfor et givet tidsrum. Dette skal formidles ved hjælp af en Android applikation. Systemet og serveren samarbejder om at kontrollerer, hvad der dispenseres til den specifikke bruger og hvornår. Denne kommunikation er ikke fuldt ud implementeret endnu. Systemet var oprindeligt tiltænkt brug i private hjem, men undervejs i processen er idéen ændret idet at fokus flyttes fra private hjem til brug i plejesektoren. 
Produktet kan på nuværende tidspunkt dosere et bestemt antal tabletter af én enkelt type. Systemet har på nuværende tidspunkt allerede implementeret muligheden for at udvide med ekstra dispenserer, og er derfor velegnet til videreudvikling. 
Til systemet er der implementeret en fungerende brugergrænseflade, som er testet med en brugervenlighedsundersøgelse, der ved adspørgelse af 15 studerende har påpeget adskillige muligheder for forbedring.
Under systemudviklingen er den største arbejdsindsats lagt i undersøgelse af teknologier og mulige løsningsforslag.
\end{abstract}

\begin{abstracten}
The report covers a third semesterproject from the Aarhus University School of Engineering. The goal is to produce a system that can function as a dispensing and sorting machine for pills. The system is intended for use in eldercare, as a way of preventing medication errors and a way of sorting pills automaticly. Furthermore, the system will be able to notify staff and relatives if the elder has not picked up their pills, within a given timeperiod. The staff and relatives will be notified via an Android application. The system and a server operate together in order to control what is dispensed to whom and when. Though this communication is not fully implemented yet. The system was originally intended for use in private homes, but during the process, the idea evolved so that the focus was changed from home use to the healthcare sector.
The product can at this time dispense a certain number of pills which will have to be of a single type. The option of adding extra dispensers is already implemented, and is therefore suitable for further development.
The system also has a functioning interface implemented, which has been tested with a user-experience study in which 15 students participated. With the feedback made several things clear, and several of these are suitable for further work.
During the system development the largest effort was put into the study of technologies and potential solutions for the problem chosen.
\end{abstracten}