%!TEX root = main.tex
\makenoidxglossaries
\newglossaryentry{gui}{
    name={GUI}, 
    description={GUI (Graphical User Interface) = Brugergrænseflade. Brugerens måde at tilgå systemet på}}

\newglossaryentry{psoc}{
    name={PSoC}, 
    description={PSoC (Programmable System-on-Chip) er et microcontroller system produceret af Cypress Semiconductor}}
\newglossaryentry{moscow}{
    name={MoSCoW}, 
    description={MoSCoW er en metode til at prioritere krav til ens system}}
\newglossaryentry{furps}{
    name={FURPS}, 
    description={FURPS er en metode til kategorisering af systemets ikke-funktionelle krav}}
\newglossaryentry{BDD}{
    name={BDD}, 
    description={BDD (Block Definition Diagram) er et diagram, som beskriver et system ved at opdele det op i mindre blokke}}
\newglossaryentry{IBD}{
    name={IBD}, 
    description={IBD (Internal Block Diagram) er et diagram, som viser forbindelserne mellem blokkene, som kan findes i et BDD}}
\newglossaryentry{pilleskuffe}{
    name={Pilleskuffe},
    description={Skuffen under dispenseren, hvor i den nederste pille ligger i. Skuffen tømmes ved at elektromagneten trækker skuffen til sig.}}
\newglossaryentry{pilledispenser}{
    name={Pilledispenser},
    description={Det rør som pillerne bliver opbevaret i.}}
\newglossaryentry{elektromagneten}{
    name={Elektromagneten},
    description={Elektromagneten er en spole der sendes strøm igennem. Når der er strøm på spolen dannes et magnetfelt. Der vil ofte være tale om det samme hvad enten der står spole eller elektromagnet.}}